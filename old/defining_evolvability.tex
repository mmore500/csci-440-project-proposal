\section{Defining Evolvability} \label{sec:defining_evolvability}

Evolvability is a principal concern to EA developers and evolutionary biologists alike. Although there are many competing definitions of evolvability in the literature, they primarily fall into one vein: the ability of a population to generate useful variation. This attribute is closely associate with ability of evolution to develop phenotypes that are well adapted to their environment. Considering evolvability, one realizes that, in the traditional static adaptive selection paradigm, the value of a solution isn't completely described by its fitness scores. Although high-scoring individuals are the ultimate goal of the EA process, during the EA process individuals that have the potential to lead to high fitness offspring being discovered by the evolutionary process are the most desirable (although they themselves might not have excellent fitness scores); the fitness score is simply a rough proxy that estimates the quality of future offspring from the  ability to innovate and adapt in evolutionary time. This observation raises a quandary: how can natural selection ``favor properties hat may prove useful to a given lineage in the future, but have no present adaptive function'' \cite{Pigliucci2008IsEvolvable}? Researchers are searching out explanations in two main areas: evolutionary selection mechanism, positing that perhaps forces beyond traditional natural selection such as divergent selection or a fluctuating fitness function might be necessary to encourage evolvability, and developmental mechanisms, positing that perhaps that indirect encoding of the phenotype adds inherent bias towards regular, modular phenotypes or allows for learned biases that canalize mutational effects towards selectively-advantageous ends.

\subsection{Evolvability as Heritable Variation}

Generating heritable variation is a prerequisite for generating useful variation.\mindmapmark{\evolvabilityheritablevariation} Focusing on the generation of heritable variation aspect of evolvability, discounting bias towards useful variation, an important theoretical distinction can be made between individual evolvability versus population evolvability. Individual evolvability refers to the potential of an individual to generate a diverse set of offspring from an individual, the ``behavioral diversity of its immediate  offspring,  and  select  organisms  with  increased offspring variation.'' \cite{Mengistu2016EvolvabilityIt}. Figure \inputandref{high_vs_low_individual_evolvability} contrasts high and low individual evolvability. A population with high individual evolvability might be seen as favoring regions of the genotype landscape that map to a highly variable set of phenotypes so that a highly diverse set of phenotypes are available to be reached via short travel from a particular point in the genotype space. In contrast, population evolvability refers to total amount of phenotypic diversity among potential offspring of a population as a whole \cite{Wilder2015ReconcilingEvolvability}. These two types of evolvability are illustrated in Figure \inputandref{individual_vs_population_evolvability}. Although individual and population evolvability might be correlated to some extent, there is no a direct relationship between the two. As Wilder et. al admonish, `population-level evolvability is not equal to the sum over individual evolvability because the novel phenotypes contributed by different individuals may be redundant.'' \cite{Wilder2015ReconcilingEvolvability} The difference between these two types of evolvability is more than semantic; it is thought that population-level evolvability is a much stronger indication of the ability of an evolutionary process to widely explore its search space, discover adaptive variability, and, ultimately, to generate highly-adapted individuals. Wilder et. al argue this point convincingly.
\begin{displayquote}
``On the one hand, evolvable individuals are more likely than others to introduce phenotypic variation in their offspring. On the other hand, in evolvable populations a greater amount of phenotypic variation is accessible to the population as a whole, regardless of how evolvable any individual may be in isolation'' \cite{Wilder2015ReconcilingEvolvability}
\end{displayquote}

\subsection{Evolvability as Useful Variation}

Evolvability can also be seen as the ability of a representation exert a bias towards generating useful variation.\mindmapmark{\evolvabilityusefulvariation} This bias might be innate to the representation, as in Figure \inputandref{direct_irregular_vs_indirect_regular}, or ``learned,'' as in Figure \inputandref{canalization_vs_noncanalization}.  Canalization is defined as the representation's ability to ``learn'' the fitness bias of an environment and control the variability generated among its offspring to exploit that bias, the ability to ``manipulate its own bias over the course of evolution'' \cite{Reisinger2007AcquiringRepresentations}. When viewing evolvability in this manner, an important distinction can be drawn between latent evolvability and acquired evolvability. According to Reisinger et al., latent evolvability describes ``the representation’s underlying capacity for becoming evolvable'' while acquired evolvability describes ``evolvability learned in response to a particular fitness function'' \cite{Reisinger2005TowardsEvolvability}.

\subsection{Quantifying Evolvability}

In light of scientific interest in evolvability, the question of how to quantify evolvability is central to being able to study it. Several measures of evolvability have been proposed. Reisinger et al. measure evolvability as the correlation between the bias towards useful variation that a representation gains and information on the structure of the varying fitness function that the representation receives. Specifically, they vary the amount of information about the way that the fitness function changes over time by controlling the generational rate at which the fitness function is changed; too fast and too slow should result in little information for the representation. Maximum bias should be observed at a middle rate of fitness function change \cite{Reisinger2005TowardsEvolvability}. Mengistu et al. measure evolvability as the number of different behaviors that are present among a sample of an individual's offspring. Behaviors are binned according to euclidian distance in the phenotype space; if the phenotypic distance between two behaviors is greater than a threshold value then those behaviors are considered distinct \cite{Mengistu2016EvolvabilityIt}. Wilder et al. take a similar approach, but extend it to consider population evolvability. They sample the possible offspring of a population and count the number of distinct phenotypes observed.

% \begin{itemize}
%   \item amount of information able to acquire (different speeds of varying fitness function)\cite{Reisinger2005TowardsEvolvability} 
%   \item RMS distance in behavioral characterization space (population diversity), ``We approximate an individual’s capacity to generate
% future phenotypic variation by measuring phenotypic
% variability among a sample of the individual’s simulated off-
% spring (which are discarded). Such variability is quantified as
% the number of unique behaviors; in particular, each offspring
% is considered sequentially and added to a list of unique behaviors
% only if its behavior is significantly different from the
% behaviors of organisms already in the list. Two behaviors are
% considered different if the distance between them according
% to a domain-specific behavioral distance metric is above a
% pre-specified threshold'' \cite{Mengistu2016EvolvabilityIt}
%   \item ``a measurement of evolvability should characterize the
% amount of variability that can be accessed in an individual or population's genetic neighborhood; number of distinct phenotypes in a genetic neighborhood around individual; amounts to Monte Carlo sampling of the phenotypic space surrounding an individual \cite{Wilder2015ReconcilingEvolvability} change
% \end{itemize}

% \cite{Reisinger2005TowardsEvolvability} \cite{Mengistu2016EvolvabilityIt} \cite{Wilder2015ReconcilingEvolvability} \cite{Tarapore2015EvolvabilityBenchmarks}

% \begin{itemize}
%   \item idea: the value of a solution isn't just its ability to generate good fitness scores, but the ability to innovate and adapt in evolutionary time $\rightarrow$ question: how could natural selection ``favor properties that may prove useful to a given lineage in the future, but have no present adaptive function''?, ``teleological fallacy'', multiple levels of selection (i.e. species selection in addition to individual selection?) \cite{Pigliucci2008IsEvolvable}

%   \item individual evolvability: ability to generate a diverse set of offspring from an individual ``behavioral diversity of its immediate  offspring,  and  select  organisms  with  increased offspring variation.'' \cite{Mengistu2016EvolvabilityIt}

%   \item population evolvability
%   \begin{itemize}
%     \item ``It is important to note that population-level evolvability is not equal to the sum over individual evolvability because the novel phenotypes contributed by different individuals may be redundant'' \cite{Wilder2015ReconcilingEvolvability}
%     \item ``On the one hand, evolvable individuals are more likely than others to introduce phenotypic variation in their offspring. On the other hand, in evolvable populations a greater amount of phenotypic variation is accessible to the population as a whole, regardless of how evolvable any individual may be in isolation'' \cite{Wilder2015ReconcilingEvolvability}
%   \end{itemize}

%   \item ability to ``learn'' bias of environment and to canalize offspring \cite{Reisinger2005TowardsEvolvability} (to ``manipulate own bias over the course of evolution'' \cite{Reisinger2006SelectingRepresentations})
%   \begin{itemize}
%     \item  ``latent evolvability will be used to describe the representation’s underlying capacity for becoming evolvable'' \cite{Reisinger2005TowardsEvolvability}
%     \item ``acquired evolvability will be used to refer to its evolvability learned in response to a particular fitness function'' \cite{Reisinger2005TowardsEvolvability}
%     \item ``the evolvability of a genome can be approximated with the fitness of the local mutation landscape around that genome'' \cite{Reisinger2007AcquiringRepresentations}
    
%   \end{itemize}
% \end{itemize}