\section{Evolvability}
Evolvability is a principal concern to Evolutionary Algorithm researchers and evolutionary biologists alike. Although many competing definitions of evolvability exist in the literature, they primarily fall into one vein: the ability of a population to generate useful variation. Breaking the concept down further, evolvability stems from:
\begin{enumerate}
\item potential for generation of heritable phenotypic variation through mutation, and \label{itm:heritable_variation}
\item bias towards the generation of potentially useful phenotypic variation through mutation.\footnote{This might also be conceptualized as a bias against lethal or otherwise deleterious mutation.} \label{itm:canalization}
\end{enumerate}

Figure \inputandref{arabidopsis_mutants}, which juxtaposes wild-type \textit{Arabidopsis thaliana} with several mutant strains, provides a biologically-motivated illustration of these concepts. The depicted individuals are clustered together in the genotypic space of \textit{Arabidopsis thaliana}. Although the genotypic distance between these phenotypes is small, a relatively broad range of phenotypic variability is represented. The rich diversity of form is hence accessible to evolutionary search allows rapid adaptation to environmental changes or novel adaptation to the existing environment. This comparison illustrates, in particular, the modularity of the system under genotype-phenotype mapping; the mutant strains exhibit significant variability in overall composition of the individual while preserving the general character of existing components. Although not depicted above, the local genotypic neighborhood also maps to a great number of phenotypes indistinguishable from wild-type individuals, many specimens that differ by more calibration-like adjustments to various phenotypic properties, as well as -- doubtlessly -- a not-insignificant (but not overwhelming) smattering of fundamentally defective phenotypes. These six specimens are, of course, insufficient evidence to substantiate a rigorous argument to this end, but the apparent preservation of modularity and the supposed moderate incidence of lethality can be cast as canalization, bias towards the generation of potentially useful variation. Figure \inputandref{canalization_example} describes a more concrete biological manifestation of canalization.

Evolvability is a question of interest not just to EA researchers, but also to evolutionary biologists as they push past the modern synthesis and grapple with the extended evolutionary synthesis. The question of evolvability poses a theoretical quandary. The traditional model of static fitness-based selection (i.e. the modern synthesis) posits that selection is performed based on compatibility between an individual's phenotype and the environment but not traits of individual related to the potential to innovate and adapt in evolutionary time.\footnote{Examples of such traits include modularity, canalization, degeneracy.} At face value, this modern synthesis fails to explain the emergence of those traits that has been observed in biological evolution -- and have been sorely lacking in attempts at digital evolution, how natural selection might  ``favor properties that may prove useful to a given lineage in the future, but have no present adaptive function'' \cite{Pigliucci2008IsEvolvable}. To resolve this dilemma researchers are considering a number of hypotheses, including but not limited to
\begin{itemize}
\item evolutionary selection mechanisms, positing that mechanisms beyond traditional static fitness-based selection (such as divergent selection or a fluctuating selection criteria) might promote evolvability \cite{Mengistu2016EvolvabilityIt, Kashtan2007VaryingEvolution};
\item developmental mechanisms, positing that perhaps that indirect encoding of the phenotype adds inherent bias towards regular, modular phenotypes \cite{Clune2011OnRegularity} or allows for the encoding of biases that canalize mutational effects towards selectively-advantageous ends \cite{Reisinger2007AcquiringRepresentations}; and
\item phenotypic plasticity, positing that environmental influence on the phenotype -- by altering the trajectory of the developmental process or otherwise inducing phenotypic changes in response to environmental stimulus \cite{Fusco2010PhenotypicConcepts} -- might promote evolvability \cite{Moczek2011TheInnovation}.
\end{itemize}
It seems likely that evolvability stems from a large and diffuse set of contributing factors. The establishment -- or rejection -- of empirical causal links between theoretical complications of the modern evolutionary synthesis and evolvability is a key research goal in the field; this type of inquiry will help determine how complicated of a model is necessary to account for evolution as observed in biological history and how complicated of a model is necessary to realize digital evolving systems with performance akin to their biological counterpart. Although scientists traditionally strive for simplicity biological systems might also be viewed from the perspective of an engineer, in which complexification in pursuit of performance is more familiar and comfortable \cite[pg 6,7]{Sterling2015PrinciplesDesign}. Naive efforts to model biological systems at an arbitrary level of detail, however, are also unlikely to be tractable or, ultimately, useful (although perhaps of interest from a certain scientific perspective) \cite[pg 354]{Downing2015IntelligenceSystems}. Thus, the establishment -- or rejection -- of empirical causal links between theoretical complications of the modern evolutionary synthesis and evolvability is a key research goal in the fields of evolutionary biology and evolutionary algorithm design. It is hoped that this line of inquiry, into which this project falls, will help determine which elaborations are necessary to account for evolution as observed in biological history and to realize digital evolution with performance akin to its biological counterpart.