\section{Miscellania}
\subsection{Challenges and Learning Opportunities}

I chose to pursue this project because it directly relates to my own career interests. I hope to study in graduate school. This design of this project builds off of research that I conducted last semester through my thesis unit. 

This experimental project poses a significant challenge -- experimental inquiry in and of itself is unpredictable, so analysis of experimental results are typically a fraught process. Building a reasonably efficient and highly accurate implementation of the necessary computational models will also be a difficult task. Finally, communicating the results of my project, its evolution-theoretic basis, and its relevance to less biologically-inclined mathematicians and computer scientists as well as a more general audience will also be hard.  

These hurdles will provide a valuable learning opportunity. The skills I develop conducting, analyzing, and presenting computational experiments will directly apply to my graduate career where I will be doing nearly the same things for (hopefully) a meager living. I am also genuinely hoping through this project to gain and share new insight that is of active interest to researchers in computer science and in evolutionary biology (as well as myself).

\subsection{Deliverable Functionality}

This project will not result in deliverable functionality in, perhaps, the traditional sense of these proposals. Although I intend to develop functional computational models of genetic regulatory networks and scripts to perform data analysis, these products will be of little use -- or even interest -- outside the scope of this project. Instead, my deliverables would best be characterized as my final presentation and report. Although not formally linked to this course, my Honors thesis document, Honors thesis presentation, and Math/CS Departmental presentation will provide additional outlets for deliverable production -- and ensure that the communication-based deliverables that are formally associated with the capstone course are well-developed.

I would characterize the ``basic'' functionality as an inconclusive or negative result. This outcome, which is typical in experimental research, is not project failure. In this case, I will prepare and present a thoughtful analysis of my experimental protocol that attempts to account for the results and recommendations for future work. The attached schedule includes several weeks for experimental adjustment, which will ensure that several attempts can be made at trying to pinpoint what is at play behind the experimental results observed and, even if a positive result is not ultimately obtained, a series of experimental follow-ups that eliminate certain hypothesis and favor others will provide interesting -- and meaningful -- material to prevent. My ``basic'' functionality will be more than ``Here's what I did. It didn't work.''

My stretch functionality would be a positive result and further experimentation and analysis to better characterize/support that result or pursuing questions raised by the positive result. However, in this case, my deliverables will not be affected dramatically; I will present a discussion of my methods, my results, the implications of my results, and recommendations for future work. 
