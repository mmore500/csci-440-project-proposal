\section{Evolutionary Algorithms}

At the genesis of modern computing, the 1950s, researchers began to apply advancing computational capabilities to investigate and test models of biological evolution. Very quickly they realized the potential of virtual evolution to achieve other ends, setting into motion a line of research that has since blossomed into the field of evolutionary algorithms (EA) design. These algorithms, which use mechanics inspired by biological evolution to evolve novel solutions to a wide array of problems, share a generally consistent basic methodology. The process begins with a population of randomly generated solutions. In a generation-based loop, an elite subset of the population is selected for their fitness (their quality as a solution), subjected to random changes, and recombined with each other to form the next generation. The cycle repeats for as many iterations as desired, and fitness tends to increase with each iteration. Figure \inputandref{working_principle} provides a graphical overview of this process.

When discussing evolutionary algorithms (as well as their biological counterpart) an important distinction is drawn between phenotype and genotype. Phenotype refers to the characteristics of an individual that interact with its environment to determine its fitness. In biology, the physical form of an organism (i.e. its body) is the phenotype. In evolutionary algorithms, the phenotype refers to the characteristics of an individual that are evaluated during selection. Genotype refers to information that is used to determine the phenotype that is passed from generation to generation. In biology, a DNA sequence serves as the genotype. Although many different genotypic encodings are employed in evolutionary algorithms, in evolutionary algorithms the genotype ultimately boils down to a collection of digital information. A glossary reviewing this vocabulary, as well as a handful of other terms, is provided in Section \ref{sec:glossary}.


Researchers and engineers have widely demonstrated the ability of EA to attack labor-intensive optimization problems and to discover novel solutions beyond the reach of human ingenuity \cite{Poli2008AProgramming}. The intervening half century of EA research has seen diversification of the general evolutionary search process described above and diversification of the contents and format of candidate solutions. Today, evolutionary algorithms serve a dual purpose: a tool for biological inquiry and an algorithm for the automatic design of solutions to problems.